\subsection{Topological Groups and Completion}

Throughout, let $G$ be a topological group.\footnote{There are two 
equivalent ways of understanding this notion. First --- the more 
classical way --- $G$ is a group together with a topology on its 
underlying set in which left/right multiplication with respect to 
any element $g \in G$ and taking inverse are continuous are 
continuous functions from $G$ to $G$.  Second --- the more hipster 
way --- is to say that $G$ is a group object in the category of 
topological spaces. That is, there are maps of topological space
\[
\mu: G \times G \to G\;\;\;\textrm{(composition) and}\;\;\;\;
\iota: G \to G\;\;\;\textrm{(inverse)}
\]
satisfying the axioms for composition and inverse.} In particular, 
since translates by elements of $g$ are homeomorphisms, the 
neighborhoods of the identity $e$ uniquely define the topology on 
$G$. Immediately, we have the following useful result:

\begin{lem}\label{lem:top_grp_neigh_props}
Let $H$ be the intersection of all open neighhoods of $G$. Then
\begin{enumerate}
\item $H$ is a normal subgroup of $G$.

\item $H$ is the closure of $\{e\}$.

\item $G/H$ is Hausdorff.

\item $G$ is Hausdorff if and only if $H = \{e\}$.
\end{enumerate}
\end{lem}

A version of this proof for an abelian group $G$ can be found as
\cite{AM} Lemma 10.1.

\begin{proof}
\begin{enumerate}
\item fix $h \in H$, then $h^{-1} \in H$ (since taking inverse is 
also a homeomorphism). In particular, translates of any open 
neighborhood by $h$ contains $e$. It follows that $hH = H$, and 
that $H$ is a subgroup. Similarly, conjugating by $g \in G$ is a 
homeomorphism, whence $gHg^{-1} = H$. Hence $H \nsubg G$.

\item Fix $U$ an open set of $G$ that does not contain $e$. We
claim that $U$ does not meet $H$. Suppose not. Then fix $h \in U 
\cap H$. But $e \in hU^{-1}$, and thus $hU^{-1}$ must be a 
neighborhood of $e$. By definition, $h \in hU^{-1}$. It follows
that $e \in U$, a contradiction.

Therefore, $H$ is closed, and any closed set containing $e$
also contains $H$. Thus, $H$ is the closure of $\{e\}$.

\item By (1), $G/H$ is also a group, with topology given by the
quotient topology. The map $G \to G/H$ is an open surjection (and 
thereby also a closed map). Since $H$ is closed, the set $\{H\}$ 
in $G/H$ is closed. Let $G' \defeq G/H$, and let us abuse 
notations and write $e$ for the coset $H$.

In particular, $G' \times G' \to G'$ given by $(g, h) \mapsto 
g^{-1}h$ is continuous, and the inverse image of the closed set 
$\{e\}$ is precisely the diagonal, which is closed. It follows 
that $G/H$ is Hausdorff.

\item If $H = \{e\}$, then $G/H = G$, and therefore, $G$ is Hausdorff
by (3). Conversely, if $G$ is Hausdorf, then for any $h \neq e$,
there exists some open set $U$ containing $e$ but not $h$. It
follows that $H = \{e\}$, as desired.
\end{enumerate}
\end{proof}

In the case where the fundamental system of neighborhoods 
$\{G_\alpha\}$ of $e$ are given by normal subgroups of $G$, then 
we have an inverse system.

\begin{defn}
Let $G$ be given as above. The \DEF{group completion}{completion}
$\Complete{G}$ of $G$ is given by the inverse limit
\[
\Complete{G} = \varprojlim_{G_\alpha \subg G} G/G_\alpha.
\]
\end{defn}

Though the concrete theory of limits and colimits will \emph{not} 
be discussed here, we give a short description of (a canonical
representation\footnote{We write this because limit/colimits are
really an isomorphism class of objects satisfying certain universal
property. The fact that $\Complete{G}$ has any description as a set
of elements is a consequence of $G$ being an object in a concrete 
category (see \cite{ML} V.2.1).} of) $\Complete{G}$. We can 
identify $\Complete{G}$ with the collection of tupple 
\[
(x_\alpha) \in \prod_{G_\alpha} G/G_{\alpha}
\]
where for every group morphism $\pi : G/G_\alpha \to G/G_\beta$, 
$\pi(x_{\alpha}) = x_\beta$. In other words, the tupple satisfies
\DEF{inverse system!coherence condition}{the coherence condition}
for the inverse system $\{G/G_\alpha\}$.

\begin{prop}\label{prop:group_compl_is_group}
The completion $\Complete{G}$ of the topological group $G$ is a 
group.
\end{prop}
\begin{proof}[Very sketchy proof.]
Let $(x_\alpha), (y_\alpha) \in \Complete{G}$.

Define $(x_\alpha y_\alpha)$ to be the composition of $(x_\alpha)$ 
and $(y_\alpha)$, and define $(x_\alpha^{-1})$ to be the inverse
of $(x_\alpha)$. Verify that both sequences satisfy the coherence
condition for $\{G/G_\alpha\}$, and define elements in 
$\Complete{G}$.

We abuse notation and write $e \in \Complete{G}$ for the sequence 
$(e \in G/G_\alpha)$. Then $e$ is also an element of $\Complete{G}$.
Together, the composition, inverse, and $e$ satisfy the axioms of
a group.
\end{proof}

We do not show why the group completion of such a topological 
group is the limit object in the category of topological space in
this section. We instead delegate this task to Appendix 
\ref{appendix:profinite}, in which the result is proved in some 
terribly gruesome details, couple with discussions of a number of
topics that have more to do with category theory than the subject
at hand. (We talk about profinite topological spaces for an entire 
chapter!) We consider the reader duly warned.

Nonetheless, we can give a short description of what the open sets
are in this case. Fix any normal subgroup $H$ of $G$. Then, by 
intersecting each open set in the system of neighborhoods 
$\{G_\alpha\}$ of $e \in G$ by $H$, we have an system of 
neighborhoods $\{G_\alpha \cap H\}$ of $e \in H$, from which we 
may define $\Complete{H}$.

\begin{prop}\label{prop:compl_of_normal_is_normal}
Let $H$ be a normal subgroup of $G$. Then $\Complete{H}$ is a 
normal subgroup of $\Complete{G}$.
\end{prop}
\begin{proof}
Using the element-based description of $\Complete{H}$, we derive
the proposition from the fact that each $H/(H \cap G_\alpha)$ (as
a subgroup of $G/G_\alpha$) is normal.

Indeed, $\Complete{H}$ consists of coherence sequences 
$(x_\alpha)$ in $\Complete{G}$ where $x_\alpha$ is in the subgroup 
$H/(H \cap G_\alpha)$ of $G/G_\alpha$. Fixing one such sequence, and
conjugate by $(y_\alpha) \in \Complete{G}$. Then the resulting 
sequence consists of the tupple of elements $y_\alpha x_\alpha 
y_\alpha^{-1} \in H/H \cap G_\alpha$ by the normality of $H/H \cap 
G_\alpha$. It follows that conjugating by $(y_\alpha)$ stabilizes
$\Complete{H}$.
\end{proof}

Applying Prop. \ref{prop:compl_of_normal_is_normal} to each open
normal subgroup in $\{G_\alpha\}$, we have a system of 
neighborhoods $\{\Complete{G_\alpha}\}$ of $e \in \Complete{G}$.
By Lemma \ref{lem:top_grp_neigh_props}, this defines the topology 
on $\Complete{G}$, and makes $\Complete{G}$ into a topological 
group. The fact that the group operations are continuous with 
respect to this topology is by more or less by construction. That 
is, a set is open in $\Complete{G}$ if and only if it is a 
translate of a neighborhood of $e$. Perhaps the only statement
with a modicum of content in verifying is that the inverse 
operation on $\Complete{G}$ is continuous, but that is more tedius 
than enlightening, and we leave its verification squarely on the
shoulder of the skeptics.

Notice that there exists a canonical map from $G$ into 
$\Complete{G}$. Using the description of $\Complete{G}$ given before
Prop. \ref{prop:group_compl_is_group}, we have the following 
description for the map. If $\pi_\alpha: G \to G/G_\alpha$, then
the injection $i: G \to \Complete{G}$ is given by
\[
i(g) = (\pi_\alpha(g)).
\]
Since $\pi_\alpha(gh) = \pi_\alpha(g)\pi_\alpha(h)$, we see that
$i$ is a group homomorphism. In fact, per the preceding discussion,
$i$ is a map of topological groups. (See Prop.
\ref{prop:compl_is_morph_of_top_grps}.)

\begin{caution}
The reader may be tempted to say that $i$ (which could stand for 
``injection'') is actually a monomorphism. This is \emph{NOT} true 
in general. Suppose $G$ is not Hausdorff. Then the image of some 
$h \in H = \bigcap_\alpha G_\alpha$ would be mapped to $e \in 
\Complete{G}$; hardly a monomorphism. 
\end{caution}

\begin{defn}
Let $G$ be a topological group with a system of neighborhoods
given by normal subgroups. We say that $G$ is a \DEF{complete 
topological group}{complete topological group} if the canonical
map from $G$ to $G/G_\alpha$ is an isomorphism of topological
groups.
\end{defn}

\begin{prop}
The map $i: G \to \Complete{G}$ is an injection if and only if
$H = \cap_\alpha G_\alpha = \{e\}$. In particular, complete 
topological groups are Hausdorff.
\end{prop}
\begin{proof}[Sketch of proof]
The fact that $x \in H$ is in the kernel is clear; fix $x$ in
the kernel. Then $h \in G_\alpha$ for every $\alpha$. The map $i: 
G \to \Complete{G}$ is an injection if and only if $H = \{e\}$. 
Now, use Lemma \ref{lem:top_grp_neigh_props} (4).
\end{proof}

To justify the name ``completion'', it is expected that 
$\Complete{G}$ is complete. This is precisely the content of Cor. 
\ref{cor:completion_is_complete}. We prove this in the following 
steps:

\begin{prop}
The canonical homomorphism $i: G \to \Complete{G}$ descends to an 
isomorphism $i_\alpha: G/G_\alpha \to 
\Complete{G}/\Complete{G_\alpha}$.
\end{prop}
\begin{proof}
Consider the composition $G \to \Complete{G} \to 
\Complete{G}/\Complete{G_\alpha}$. The kernel is precisely 
$G_\alpha$, and therefore, the map factors through $G/G_\alpha$.
In particular, $G/G_\alpha$ is an injection, and it suffices to
show that this map is a surjection.

For this, fix $x_\alpha \in G/G_\alpha$, and let $x$ be a coset
representative of $x_\alpha$. Suppose $(y_\beta)$ is element of
$\Complete{G}$ whose $G/G_\alpha$ component is $x_\alpha$, then
we claim that the constant sequence $(x)$ and $(y_\beta)$ differs
by an element of $\Complete{G_\alpha}$. That is, $(x^{-1}y_\beta)$
is an element of $\Complete{G_\alpha}$.

This follows from the coherence condition. Fix $G_\beta$. We wish
to show that $x^{-1}y_{\beta} \in G/G_\beta$ lies in the image of
$G_\alpha$. Notice that $x^{-1}y_\beta$ is the image of 
$x^{-1}y_{\beta'} \in G/G_{\beta'}$ for $G_{\beta'} = G_\alpha 
\cap G_\beta$. Replacing $\beta$ by $\beta'$, we may assume that 
$G_\beta$ is a subset of $G_\alpha$. However, in this case, the 
image of $x^{-1}y_{\beta}$ under $G/G_{\beta} \to G/G_{\alpha}$ is 
$e$. The claim follows.
\end{proof}

\begin{prop}\label{prop:compl_is_morph_of_top_grps}
The map $i:G \to \Complete{G}$ is continuous. Specifically, $i$
is a morphism of topological groups.
\end{prop}
\begin{proof}
Let $U$ be an open set in $\Complete{G}$. We show that $i^{-1}(U)$ 
is open. By previous proposition, $U = i(x) + \Complete{G_\alpha}$
for some $x \in G$. Therefore, it suffices to assume that $U$ is
a neighborhood of $e \in \Complete{G}$. That is, $U = 
\Complete{G_\alpha}$. But $i^{-1}(U) = G_\alpha$, which is open in
$G$. The proposition follows.
\end{proof}

\begin{cor}\label{cor:completion_is_complete}
The completion $\Complete{G}$ is complete.
\end{cor}
\begin{proof}
There is a one-to-one correspondence of neighborhoods of $e \in G$ 
and neighborhoods of $e \in \Complete{G}$. Furthermore, 
$G/G_\alpha \iso \Complete{G}/\Complete{G_\alpha}$. Taking limits,
we have that $\Complete{G} \iso \Complete{(\Complete{G})}$.
\end{proof}


