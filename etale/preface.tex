\Preface

First, the disclaimer: The author intends to do no harm to the
mathematical society as a whole by fillings its already ample
supply of texts on this subject with yet another text. A thick
one, no less! Second, a concession. When the author tried to
learn the very same material, the author had a very tough time.
For one, the material is scattered throughout the universe of
mathematics, and without necessarily a good reference book or
even a centralizing text around which a relatively few auxiliary
text revolved, it is very time consuming to piece together the 
material from the great expanse.

Then there came the stack project \cite{Stack}, which is (and 
remains to be) the great encyclopaedia of algebraic geometry 
culminating in the development of the theory of algebraic stacks. 
This more than 2,000 page tome contains an extremely complete 
exposition of \'etale theory, but its catacombs of information is 
difficult to navigate. In these notes, the author had hoped 
desperately to form coherent ideas around the subject of \'etale 
cohomology and its applications that are as self-contained as 
possible.

It \emph{may not} be possible to have a relatively succinct text 
that is self-contained and yet highly readable --- something that 
became clear at the very beginning of this venture, but a gap can 
still be filled for those who wish to learn the subject having 
braved the first three chapters of Hartshorne's celebrated text 
(or its equivalent).

These notes came at the end of the author's graduate career,
as a distraction from his duties to his thesis. There are three
main goals to writing them. First and foremost, they are intended
to be as humorous as possible; so as to make the subject more
memorable; there are now choice phrases peppered throughout 
these pages that can be repeated amidst the graduate ranks. 
Unfortunately, more genius went into coming up with these 
sentences than the actual mathematics; the author, thereby, wishes 
to apologize for this obvious short-coming.

The second goal is to serve as a reference for a number of 
graduate students at the time who wish to learn the subject. This
is a modest goal, and in-so-far as this goal is concerned, it
is met, however satisfactorily. On this matter, the author wishes
to thank those experiments who contributed noisily to this
text. This book will forever be dedicated to them.

The third main goal is to provide an answer to this anecdote. At 
one point in the author's final year in graduate school, he was 
asked to demonstrate a few key facts in \'etale cohomology to a 
few students whose primary subject was not quite algebraic 
geometry. Having read Hartshorne and heard a lot of buzz in the 
office about \'etale theory, they were very keen to form a study 
group to tackle the material. Spirits were high. Unfortunately,
two weeks into the ``seminar'', the key players of this group 
decided that one should best leave algebraic geometry to the 
Europeans, whose avante garde attitude and lengthy doctorate 
process would be better suited for its studies. What happened? 
For one, it \emph{felt} like the material was too much, and part 
of the problem was that \'etale theory drew from so many seemingly 
disparate points in mathematics. The second point was that, there 
was not yet a suitable text the bridged the gap that did not also 
require blindly doing all the exercises of Hartshorne.

The truth is, \'etale theory is not any harder than scheme and
sheaf theory; it definitely is not more difficult than algebraic
topology. Its source material was just more scattered, that it
felt more significant. Hence the drive to compile notes, complete 
with proofs and exercises that motivates.

We develop basic theories of \'etale cohomology theory. One can
view these notes as a verbose companion to Milne's ``\'Etale
Cohomology''.
