These notes cover basic theory of valuation.

\begin{defn}
Let $G$ be a multiplicative abelian group. We say that an 
\emph{ordering} is defined in $G$ if we are given a 
multiplicatively closed subset $P$ not including $1$ such that 
$G$ is the disjoint union:
\[
P \dU P^{-1} \dU \{1\}.
\]
In this case, we call $P$ the \emph{positive elements} of $G$.
\end{defn}

For an ordering on $G$, we say that $g < h$ if $hg^{-1} \in P$.
It is easy to see that $1 < h$ if and only if $h \in P$. The
following are clear consequences:

\begin{prop}
Let $G$ be a group with an ordering by $P$. Then:

\begin{enumerate}
\item For $g, h \in G$, either $g < h$, $g = h$, or $h < g$.

\item If $g < h$ then $gg' < hg'$ for any $g' \in G$.

\item $g < h$ and $h < k$ implies that $g < k$.
\end{enumerate}
\end{prop}
