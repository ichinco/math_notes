These notes cover basic theory of valuation.

\section{General Valuation Theory}

\begin{defn}
Let $G$ be a multiplicative abelian group. We say that an 
\emph{ordering} is defined in $G$ if we are given a 
multiplicatively closed subset $P$ not including $1$ such that 
$G$ is the disjoint union:
\[
P \dU P^{-1} \dU \{1\}.
\]
In this case, we call $P$ the \emph{positive elements} of $G$.
\end{defn}

For an ordering on $G$, we say that $g < h$ if $hg^{-1} \in P$.
It is easy to see that $1 < h$ if and only if $h \in P$. The
following are clear consequences:

\begin{prop}
Let $G$ be a group as above with an ordering by $P$. Then:

\begin{enumerate}
\item For $g, h \in G$, either $g < h$, $g = h$, or $h < g$.

\item If $g < h$ then $gg' < hg'$ for any $g' \in G$.

\item $g < h$ and $h < k$ implies that $g < k$.
\end{enumerate}
\end{prop}

In fact, the above characteristics defines an ordering (though 
this will not be used in the sequel). More precisely:

\begin{prop}
Let $P$ be any subset of $G$. Define $h < g$ if $gh^{-1} 
\in P$. If ``$<$'' satisfies (1 - 3) of the above proposition, 
then $P$ defines an ordering of $G$.
\end{prop}
\begin{proof}
To show that $P$ is closed under multiplication, note that for
$g, h \in P$, $1 < g < gh$, and by (3), $1 < gh$, and therefore,
$gh \in P$. 

The fact that $G = P \dU P^{-1} \dU \{1\}$ is a consequence of
(1): fix $g \in G$, $g < 1, g = 1$ or $g > 1$ suggest precisely
that $P$ is disjoint from $P^{-1}$ and $\{1\}$, and $g \in P$,
$g = 1$ or $g \in P^{-1}$.
\end{proof}

\begin{rmk}
Suppose $G$ is any group with an ordering, then for any $g \in G$
such that $g^n = 1$ for any $n \in \Z$, then $g = 1$. Indeed, 
assume \wlog that $g \in P$ to obtain a contradiction. Then $g^n 
\in P$ for any $n$; this contradicts the fact that $1 \notin P$.

In particular, \emph{the map $g \mapsto g^n$ is injective}.
\end{rmk}

\begin{defn}
Let $K$ be a field. A \emph{valuation} of $K$ is an association 
$v: K \to G \dU \{-\infty\}$ for some $G$ with an ordering (by 
$P$, say) and an extra symbol $-\infty$
\begin{enumerate}
\item[\textbf{V1.}] $v(x) = -\infty$ if and only if $x = 0 \in K$,

\item[\textbf{V2.}] $v(xy) = v(x)v(y)$ for all $x, y \in K$,

\item[\textbf{V3.}] $v(x + y) \leq \max(v(x), v(y))$
\end{enumerate}
\end{defn}

\begin{rmk}
It is easy to see that a valuation $v: K \to G\dU \{-\infty\}$
defines group homomorphism from $K^*$ to $G$. Noting \textbf{V2.},
we need only to verify that $v(1) = 1$. To see this, note that 
$v(1) = v(1 \cdot 1) = v(1)v(1)$. It follows that $v(1) = 
v(1)^{-1}v(1) = 1$ as desired. It follows, in particular, that
$v(x^{-1}) = v(x)^{-1}$.
\end{rmk}

We say that a valuation $v$ is \emph{trivial} if $v(K^*) = 1$.

\begin{prop}
Let $K$ be a field, and $R$ is a subring of $K$ for which $x \in 
R$ or $x^{-1} \in R$. Then $R$ is a local ring.
\end{prop}
\begin{proof}
We show that the set $\ideal{m}$ of nonunits of $R$ is an ideal 
(hence maximal). It is clear that if $x \in R$ is not a unit, then
$rx$ is not a unit for any $r \in R$.

Suppose $x, y \in \ideal{m}$ are non-units, then assume \wlog that 
$x/y \in R$ (or else $y/x \in R$). Then $1 + x/y \in R$. But
\[
1 + x/y = (x + y)/y.
\]
If $x + y$ is a unit, then $y$ is a unit, contradicting the 
assumption that $y \in \ideal{m}$. It follows that $\ideal{m}$
is an ideal. Maximality follows from the fact that any ideal
properly containing $\ideal{m}$ must in fact be unital.
\end{proof}
