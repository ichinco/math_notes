\section{Filtrations and completions.}

We begin the discussion with a topological excursion, as to lay
the ground work for discussions regarding profinite groups in
Appendix \ref{appendix:profinite}. Throughout, let $G$ be a
topological group.\footnote{There are two equivalent ways of 
understanding this notion. First --- the more classical way --- 
$G$ is a group together with a topology on its underlying set in 
which left/right multiplication with respect to any element $g 
\in G$ and taking inverse are continuous are continuous functions 
from $G$ to $G$. Second --- the more hipster way --- is to say 
that $G$ is a group object in the category of topological spaces. 
That is, there are maps of topological space
\[
\mu: G \times G \to G\;\;\;\textrm{(composition) and   }
\iota: G \to G\;\;\;\textrm{(inverse)}
\]
satisfying the axioms for composition and inverse.} In particular,
$G \times G \to G$ given by $(g, h) \mapsto g^{-1}h$ is 
continuous, and the inverse image of the closed set $\{e\}$ is 
precisely the diagonal. It follows that $G$ is Hausdorff.

\begin{defn}

\end{defn}

\begin{thm}{Artin-Rees Lemma}\label{thm:artin_rees_lemma}
Let $R$ be a Noetherian ring, $\ideal{a}$ an $R$-ideal, and $M$ a
finitely generated $R$-module, and $F^*M$ a stable 
$\ideal{a}$-filtration of $M$. If $M'$ is a submodule of $M$ then
$(M' \cap F^nM)$ is a stable $\ideal{a}$-filtration of $M'$.
\end{thm}
